\documentclass{bayeshyp}

\usepackage[logonly]{trace}% for debugging purposes only
\usepackage{lipsum}% for testing purposes only
\usepackage[math]{blindtext}% for testing purposes only

% FIXME hackish workaround that should be fixed upstream in Tufte-LaTeX
% Set up the spacing using fontspec features
\usepackage{ifxetex}
\ifxetex
  \renewcommand\allcapsspacing[1]{{\addfontfeature{LetterSpace=15}#1}}
  \renewcommand\smallcapsspacing[1]{{\addfontfeature{LetterSpace=10}#1}}
\fi

\title{Bayesian Hypothesis Testing Without Tears}
\subtitle{Examples for Students}
\authors{%
  Eric-Jan Wagenmakers,
  Michael D. Lee,
  Jeffrey N. Rouder,
  Dora Matzke,
  Maarten Marsman,
  Jonathon Love,
  Alexander Ly,
  Ravi Selkser,
  Tahira Jamil,
  Quentin F. Gronau,
  Richard D. Morey
}
\publisher{Jasp Publishing}

% FIXME Temporary
\let\subsubsection\subsection

\usepackage{bayeshypdoc}

\begin{document}

\maketitle

\tableofcontents

\mainmatter

\setcounter{secnumdepth}{1}

\part{Documentation}

\chapter{Features of \texttt{bayeshyp.cls}}

\section{Title page}

The title page has a new design and a couple new macros to go along with it.

\paragraph{Subtitle}

A subtitle may be specified using the \syn|\subtitle| macro:

\begin{syntax}
  \subtitle{Examples for Students}
\end{syntax}

\paragraph{Multiple authors}

A comma-separated list of authors may be provided using the \syn|\authors| macro. For example:
\begin{syntax}
  \authors{
    Eric-Jan Wagenmakers,
    Michael D. Lee,
    Jeffrey N. Rouder,
    Dora Matzke,
    Maarten Marsman,
    Jonathon Love,
    Alexander Ly,
    Ravi Selkser,
    Tahira Jamil,
    Quentin F. Gronau,
    Richard D. Morey
  }
\end{syntax}

The authors are listed on the title page in the order provided.
Author names will not be split across lines.

\section{Headings}

Some headings have been modified to provide more typesetting options than usual. These modifications were done at the cost of backward-compatibility, however.

The original heading commands general follow the following format:
\begin{syntax}
  \chapter[<toc entry>]{Chapter title}
\end{syntax}

The new format requires that \syn|<toc entry>| be assigned to the new \syn|toc| key. The optional argument now contains a list of key--value pairs:
\begin{syntax}
  \chapter[<key-value pairs>]{Chapter title}
\end{syntax}

The key--value pairs differ based on the heading level.

\begin{itemize}
  \item \syn|toc| \textit{(part, chapter)} This is the text you wish to appear
    in the table of contents for this heading. If left unspecified, it defaults
    to the full text of the heading. This option is useful if your heading
    contains special formatting that needs to be adjusted for typesetting in
    the table of contents or if the heading is very long and you want a shorter
    heading for the table of contents. For chapter headings, this entry is also
    used for the running heads at the top of the pages.
  \item \syn|image| \textit{(part)} On part pages, an image may be included.
    This option specifies the filename for the image.
  \item \syn|image-otions| \textit{(part)} This option specifies any
    key--value options that should be passed to the \syn|\includegraphics|
    macro used to insert the image on the part page. See the \texttt{graphics}
    package documentation for details.
  \item \syn|byline| \textit{(chapter)} Chapter headings may include a byline
    to list a chapter author, for example. The byline is not included in the
    table of contents entry. Example: \syn|\chapter[byline={with Kevin Godby}]{Chapter heading}|.
\end{itemize}




\section{Mathematical notation shortcuts}

\paragraph{Conditional probabilities}
To typeset probabilities or conditional probabilities in math mode, use the \syn|\p| macro:
\begin{syntax}
  \p(<variable>)
  \p(<hypothesis>|<evidence>)
  \p*(<hypothesis>|<evidence>)
  \p[<size>](<hypothesis>|<evidence>)
\end{syntax}
where \syn|<variable>|, \syn|<hypothesis>|, and \syn|<evidence>| are mathematical expressions or variables.
The \syn|\p*| variant will autoscale the parentheses and vertical bar to fit the contents of the expressions.
You can specify a particular size using the \syn|\p[<size>]| variant.
Possible sizes are \syn|\big|, \syn|\Big|, \syn|\bigg|, and \syn|\Bigg|.
Note also that the arguments to the \syn|\p| macro are delimited by parentheses and not braces.

\begin{gather*}
  \p(x) \quad \p(x|y) \quad \p(\text{has cancer} | \text{tested positive}) \\
  \p*(\frac{1+2}{3+4} | \sqrt{\frac{1}{1+\frac{1}{1+\frac{1}{1+\frac{1}{2}}}}}) \\
  \p[\bigg](\frac{1+2}{3+4} | \sqrt{\frac{1}{2}})
\end{gather*}


\section{Matlab code listings}

The \texttt{matlab} environment is used to display Matlab code listings.

\begin{syntax}
  \begin{matlab}[<additional settings>]
    % Matlab code goes here...

    disp('Look, a margin note!'); (*@ This comment appears in the margin note area... @*)
  \end{matlab}
\end{syntax}

Additional key--value settings for the environment may be passed as an optional argument.
Text between \syn|(*@| and \syn|*@)| will be typeset as a margin note.

\begin{matlab}
% This code listing is for Matlab code.
% This is a regular Matlab code comment.

disp('Look, a margin note!'); (*@ This comment appears in the margin note area. It may contain \LaTeX{} formatting commands and math: \[ \frac{ \p(A|B) = \p(B|A) \p(A)}{\p(B)}\text{.} \] @*)

% And some more random Matlab code just to pad things out a bit more...

\end{matlab}


%
%
% Tests
%
%

\part{Demonstrations}

\chapter{Chapter heading}
\section{Section heading}
\subsection{Subsection heading}
\paragraph{Paragraph heading} \ldots

\chapter[byline={with Kevin}]{Chapter heading}
\section{Section heading}
\subsection{Subsection heading}
\paragraph{Paragraph heading} \ldots


\Blinddocument

\part[image={example-image-a}]{Graphical Practice}

\part[image={example-image-b},image-options={width=0.25\textwidth},toc={Shorter part title}]{Super long part title that wouldn't fit nicely in a table of contents or running head}
\chapter{Chapter heading}
\section{Section heading}
\section{Another section heading---this one longer to force wrapping in the table of contents}
\lipsum

\part[toc={Old-style, short title=this}]{Another super long part title that wouldn't fit nicely in a table of contents or running head}

\chapter{Graphical Integrity}
\setcounter{page}{53}

\chapter{Sources of Graphical Integrity and Sophistication}
\setcounter{page}{79}

\part{Theory of Data Graphics}

\chapter{Data-Ink and Graphical Redesign}
\setcounter{page}{91}

\chapter{Chartjunk: Vibrations, Grids, and Ducks}
\setcounter{page}{107}

\chapter{Data-Ink Maximization and Graphical Design}
\setcounter{page}{123}

\chapter{Multifunctioning Graphical Elements}
\setcounter{page}{139}

\chapter{Data Density and Small Multiples}
\setcounter{page}{161}

\chapter{Aesthetics and Technique in Data Graphical Design}
\setcounter{page}{177}

\chapter{An Incredibly Lengthy Title to Force the Table of Contents Entry to Wrap onto a Second Line}
\setcounter{page}{181}

\appendix

\chapter{Just an Appendix}
\setcounter{page}{183}

\chapter{And Another}
\setcounter{page}{185}

\backmatter
\chapter{Epilogue: Designs for the Display of Information}
\setcounter{page}{191}
\end{document}

